\documentclass[10pt]{article}

\usepackage[utf8]{inputenc}
\usepackage[russian]{babel}

\textheight=24cm %высота текста
\textwidth=16cm %длина текста
\oddsidemargin=0pt %отступ от левого края
\topmargin=-1.5cm %отступ от верхнего края
\flushbottom %выравнивание высоты страниц

\usepackage{amsmath}
\usepackage{textcomp}


% Title Page
\title{Тестирование программ на QR разложение матрицами Хаусхолдера}
\author{}
\date{}

\begin{document}
\maketitle

\section{Программа на C}
Вычисление времени выполнения и ошибки $\frac{||QR - A||_F}{||A||_F}$ для случайно-сгенерированной матрицы $A$:
\begin{center}
    \begin{tabular}{c|c|c}
      Размерность N & Время выполнения(сек)& Точность \\
      \hline 
      256 & 0.151295 & $0.206673 * 10^{-14}$ \\
      \hline
      512 & 1.362996 & $0.325643 * 10^{-14}$ \\
      \hline
      1024 & 13.198033 & $0.332653 * 10^{-14}$ \\
      \hline
      2048 & 108.055010 & $0.442351 * 10^{-14}$ \\
    \end{tabular}
\end{center}

Если скомпилировать программу на gcc с флагом отпимизации -O3, то получим такой результат:
\begin{center}
    \begin{tabular}{c|c|c}
      Размерность N & Время выполнения(сек) & Точность \\
      \hline
      256 & 0.034629 & $0.159191 * 10^{-14}$ \\
      \hline
      512 &  0.288620 & $0.357033 * 10^{-14}$ \\
      \hline
      1024 & 4.677073 & $0.798845 * 10^{-14}$ \\
      \hline
      2048 & 44.991274 & $0.577649 * 10^{-14}$ \\
      \\
    \end{tabular}
\end{center}

Вычисление ошибки $\frac{||QR - A||_F}{||A||_F}$ для случайно-сгенерированной матрицы $A_k$
вида
$$ A_k = H_1 diag(\sigma_1, ..., \sigma_n) H_2$$
где $H_1$, $H_2$ - случайные матрицы Хаусхолдера, и $\frac{\sigma_1}{\sigma_n} = 10^k$, причем 
$\sigma_1 > \sigma_2 > ... \sigma_{n-1} > \sigma_n$:
\begin{center}
    \begin{tabular}{c|c|c}
      Размерность N & Значение k & Точность \\
      \hline
      1024 & 0 & $0.397757 * 10^{-14}$ \\
      \hline
      1024 & 4 & $0.786147 * 10^{-14}$ \\
      \hline
      1024 & 8 & $0.801663 * 10^{-14}$ \\
      \hline 
      1024 & 12 & $0.758398 * 10^{-14}$ \\
      \hline
      1024 & 16 & $0.765435 * 10^{-14}$ \\
      \hline
      1024 & 20 & $0.681711 * 10^{-14}$ \\
    \end{tabular}
\end{center}

Программа переписана с использованием библиотеки BLAS, сборка на компиляторе gcc:
\begin{center}
    \begin{tabular}{c|c|c}
      Размерность N & Время выполнения(сек) & Точность \\
      \hline
      256 & 0.029024 & $0.15 * 10^{-14}$ \\
      \hline
      512 &  0.254628 & $0.21 * 10^{-14}$ \\
      \hline
      1024 & 2.382585 & $0.35 * 10^{-14}$ \\
      \hline
      2048 & 20.111016 & $0.56 * 10^{-14}$ \\
      \\
    \end{tabular}
\end{center}

\section{Программа на Fortran}
Вычисление времени выполнения и ошибки $\frac{||QR - A||_F}{||A||_F}$ для случайно-сгенерированной матрицы $A$:
\begin{center}
    \begin{tabular}{c|c|c}
      Размерность N & Время выполнения(сек)& Точность \\
      \hline 
      256 & 0.144997 & $0.290257 * 10^{-14}$ \\
      \hline
      512 &  1.164560 & $0.212471 * 10^{-14}$ \\
      \hline
      1024 & 13.105832 & $0.313008 * 10^{-14}$ \\
      \hline
      2048 & 115.974884 & $0.649089 * 10^{-14}$ \\
    \end{tabular}
\end{center}

Если скомпилировать программу на gfortran с флагом отпимизации -O3, то получим такой результат:
\begin{center}
    \begin{tabular}{c|c|c}
      Размерность N & Время выполнения(сек) & Точность \\
      \hline
      256 & 0.025870 & $0.159191 * 10^{-14}$ \\
      \hline
      512 &  0.218357 & $0.247889 * 10^{-14}$ \\
      \hline
      1024 & 3.211895 & $0.310742 * 10^{-14}$ \\
      \hline
      2048 & 31.507637  & $0. 510653 * 10^{-14}$ \\
      \\
    \end{tabular}
\end{center}

Вычисление ошибки $\frac{||QR - A||_F}{||A||_F}$ для случайно-сгенерированной матрицы $A_k$
вида
$$ A_k = H_1 diag(\sigma_1, ..., \sigma_n) H_2$$
где $H_1$, $H_2$ - случайные матрицы Хаусхолдера, и $\frac{\sigma_1}{\sigma_n} = 10^k$, причем 
$\sigma_1 > \sigma_2 > ... \sigma_{n-1} > \sigma_n$:
\begin{center}
    \begin{tabular}{c|c|c}
      Размерность N & Значение k & Точность \\
      \hline
      1024 & 0 & $0.400560 * 10^{-14}$ \\
      \hline
      1024 & 4 & $0.772869 * 10^{-14}$ \\
      \hline
      1024 & 8 & $0.747257 * 10^{-14}$ \\
      \hline 
      1024 & 12 & $0.838277 * 10^{-14}$ \\
      \hline
      1024 & 16 & $0.887717 * 10^{-14}$ \\
      \hline
      1024 & 20 & $1.013665 * 10^{-14}$ \\
    \end{tabular}
\end{center}

После замены циклов на работу со срезами, без опции оптимизации:
\begin{center}
    \begin{tabular}{c|c|c}
      Размерность N & Время выполнения(сек) & Точность \\
      \hline
      256 & 0.008615 & $0.200607 * 10^{-14}$ \\
      \hline
      512 &  0.759589 & $0.420117 * 10^{-14}$ \\
      \hline
      1024 & 6.479639 & $0.389186 * 10^{-14}$ \\
      \hline
      2048 & 51.728363  & $0.668168 * 10^{-14}$ \\
      \\
    \end{tabular}
\end{center}

После замены циклов на работу со срезами, с опцией оптимизации -O3:
\begin{center}
    \begin{tabular}{c|c|c}
      Размерность N & Время выполнения(сек) & Точность \\
      \hline
      256 & 0.0072361 & $0.159900 * 10^{-14}$ \\
      \hline
      512 &  0.627861 & $0.348846 * 10^{-14}$ \\
      \hline
      1024 & 6.162336 & $0.324858 * 10^{-14}$ \\
      \hline
      2048 & 50.556617  & $0.531992 * 10^{-14}$ \\
      \\
    \end{tabular}
\end{center}

Программа переписана с использованием библиотеки BLAS, сборка на компиляторе gfortran:
\begin{center}
    \begin{tabular}{c|c|c}
      Размерность N & Время выполнения(сек) & Точность \\
      \hline
      256 & 0.0027562 & $0.159116 * 10^{-14}$ \\
      \hline
      512 &  0.235498 & $0.231419 * 10^{-14}$ \\
      \hline
      1024 & 2.441742 & $0.379806 * 10^{-14}$ \\
      \hline
      2048 & 20.028060  & $0.422820 * 10^{-14}$ \\
      \\
    \end{tabular}
\end{center}

\section{Lapack и mkl}

Компиляция на icc и gfort с ключом -mkl=sequential и -O3. Программа написанная на Blas Vs программа напианная на Lapack.
\begin{center}
    \begin{tabular}{c|c|c|c}
      Язык & Lapack/Blas & Размерность N & Время выполнения(сек) \\
      \hline
      C & Lapack & 2048 & 0.64 \\
      \hline
      C &  Blas & 2048 & 10.65 \\
      \hline
      Fortran & Lapack & 2048 & 0.65 \\
      \hline
      Fortran &  Blas & 2048 &  10.16 \\
      \hline
    \end{tabular}
\end{center}

\end{document}          
